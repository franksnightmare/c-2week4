\documentclass[11pt]{article}

\usepackage{times}
\usepackage[english]{babel}

% -----------------------------------------------
% especially use this for you code
% -----------------------------------------------

\usepackage{courier}
\usepackage{listings}
\usepackage{color}
\usepackage{tabularx}
\usepackage{graphicx}

\definecolor{Gray}{gray}{0.95}

\definecolor{mygreen}{rgb}{0,0.6,0}
\definecolor{mygray}{rgb}{0.5,0.5,0.5}
\definecolor{mymauve}{rgb}{0.58,0,0.82}

\lstset{language=C++,
	basicstyle = \normalsize\ttfamily,   % the size and fonts that are used
	tabsize = 2,                    % sets default tabsize
	breaklines = true,              % sets automatic line breaking
	keywordstyle=\color{blue}\ttfamily,
	stringstyle=\color{red}\ttfamily,
	commentstyle=\color{mygreen}\ttfamily,
	numbers=left,
	keepspaces=true,
	showspaces=false,
	showstringspaces=false,
}

\begin{document}

\title{Programming in C/C++ \\
       Exercises set four: containers
}
\date{\today}
\author{Christiaan Steenkist \\
Jaime Betancor Valado \\
Remco Bos \\
}

\maketitle

\section*{Exercise 22, Containers solving complex tasks}
We are asked to order all words obtain by the standard input and print them in the screen.

\subsection*{Code listings}
\lstinputlisting[caption = main.cc]{src/a22/main.cc}


\section*{Exercise 23, vectors and shrinking}
So we experimented with slicing off extra capacity with vectors and a class with a vector as a data member.

\subsection*{Output}
\begin{lstlisting}
size: 10 capacity: 16
size: 11 capacity: 16
size: 11 capacity: 11

size: 11 capacity: 16
size: 12 capacity: 16
size: 12 capacity: 12
\end{lstlisting}

\subsection*{Code listings}
\lstinputlisting[caption = main.ih]{src/a23/main.ih}
\lstinputlisting[caption = main.h]{src/a23/main.h}
\lstinputlisting[caption = main.cc]{src/a23/main.cc}
\lstinputlisting[caption = printer1.cc]{src/a23/printer1.cc}
\lstinputlisting[caption = printer2.cc]{src/a23/printer2.cc}
\lstinputlisting[caption = reader.cc]{src/a23/reader.cc}

\subsubsection*{UniqueWordList}
\lstinputlisting[caption = uniquewordlist.ih]{src/a23/uwl/uniquewordlist.ih}
\lstinputlisting[caption = uniquewordlist.h]{src/a23/uwl/uniquewordlist.h}
\lstinputlisting[caption = addword.cc]{src/a23/uwl/addword.cc}
\lstinputlisting[caption = capacity.cc]{src/a23/uwl/capacity.cc}
\lstinputlisting[caption = capacityconst.cc]{src/a23/uwl/capacityconst.cc}
\lstinputlisting[caption = findword.cc]{src/a23/uwl/findword.cc}
\lstinputlisting[caption = \texttt{operator=}.cc]{src/a23/uwl/operator=.cc}
\lstinputlisting[caption = size.cc]{src/a23/uwl/size.cc}
\lstinputlisting[caption = sizeconst.cc]{src/a23/uwl/sizeconst.cc}
\lstinputlisting[caption = swap.cc]{src/a23/uwl/swap.cc}


\section*{Exercise 24, Containers solving complex tasks}
Now, we are asked to count the number of repetitions of each word, this is a continuation from exercise 22.

\subsection*{Code listings}
\lstinputlisting[caption = main.cc]{src/a24/main.cc}


\section*{Exercise 25, unique keys}
We made a snippet of code to count the number of unique keys in an \texttt{unordered\_multimap}.
Never again.

\subsection*{Code listings}
\lstinputlisting[caption = main.cc]{src/a25/main.cc}

\section*{Exercise 26, signal handling}
We made the class interface for the Signal class and made a TestHandler class that inherits from the class SignalHandler.

\subsection*{Code listings}
\lstinputlisting[caption = signal.h]{src/a2627/Signal/signal.ih}
\lstinputlisting[caption = signal.h]{src/a2627/Signal/signal.h}
\lstinputlisting[caption = signalhandler.ih]{src/a2627/SignalHandler/signalhandler.ih}
\lstinputlisting[caption = signalhandler.h]{src/a2627/SignalHandler/signalhandler.h}
\lstinputlisting[caption = testhandler.h]{src/a2627/SignalHandler/testhandler.h}
\lstinputlisting[caption = testhandler.cc]{src/a2627/SignalHandler/testhandler.cc}
\lstinputlisting[caption = destructor testhandler.cc]{src/a2627/SignalHandler/detesthandler.cc} 

\section*{Exercise 27, implementing singleton functionality}
We have implemented the member function that belong to the singleton property of the class Signal.

\subsection*{Code listings}
\lstinputlisting[caption = instance.cc]{src/a2627/Signal/instance.cc}
\lstinputlisting[caption = destructor of signal]{src/a2627/Signal/designal.cc}

\end{document}
